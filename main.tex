% This is samplepaper.tex, a sample chapter demonstrating the
% LLNCS macro package for Springer Computer Science proceedings;
% Version 2.20 of 2017/10/04
%
\documentclass[runningheads]{llncs}
%
\usepackage{graphicx}
% Used for displaying a sample figure. If possible, figure files should
% be included in EPS format.
%
% If you use the Hyperref package, please uncomment the following line
% to display URLs in blue roman font according to Springer's eBook style:
% \renewcommand\UrlFont{\color{blue}\rmfamily}

\usepackage{orcidlink} % Orcid links
% Fix underscore in dois
\usepackage[strings]{underscore}

% Listing settings
\usepackage{listings}
\usepackage{xcolor}

\lstdefinelanguage{Interaction}[]{Java}{
    comment=[l]{---},
}

\lstset{emph={  
    synchronize
    },emphstyle={\color{blue}\bfseries}
}

\begin{document}
% https://www.discotec.org/2024/coordination
% Regular papers 7-15 pages + references
% Artifacts using Zenodo together with this repo. Maybe a big CSV file with our classification results is input for some plots (clustering). Then, the source code for that should also be added.
\title{Classifying coordination approaches}
%
%\titlerunning{Abbreviated paper title}
% If the paper title is too long for the running head, you can set
% an abbreviated paper title here
%
\author{Tim Kr\"{a}uter\inst{1}\orcidlink{0000-0003-1795-0611} \and
Julien Deantoni\inst{2}\orcidlink{0000-0001-6962-7846}
Adrian Rutle\inst{1}\orcidlink{0000-0002-4158-1644} \and
Harald K\"{o}nig\inst{3,1}\orcidlink{0000-0001-6304-6311} \and
Yngve Lamo\inst{1}\orcidlink{0000-0001-9196-1779}}
%
\authorrunning{T. Kräuter et al.}
% First names are abbreviated in the running head.
% If there are more than two authors, 'et al.' is used.
\institute{Western Norway University of Applied Sciences, Bergen, Norway 
\email{tkra@hvl.no, aru@hvl.no, yla@hvl.no} \and
University Cote d’Azur, Sophia Antipolis, France \\
\email{julien.deantoni@univ-cotedazur.fr} \and
University of Applied Sciences, FHDW, Hanover, Germany\\
\email{harald.koenig@fhdw.de}}
%
\maketitle              % typeset the header of the contribution
%
\begin{abstract}
TODO
\keywords{
ADL \and
Coordination language \and
Coordination framework \and
Co-Simulation \and
Feature model
}
\end{abstract}

\section{Introduction}

% Categorizing ADLs, Coordination languages, Co-simulation, and Coordination frameworks (BCOOL and me).

% Paper outline

\section{Preliminaries} \label{sec:preliminaries}

% Define ADL and give examples.
% Cite ADLs and a survey about them.

% Define Coordination language and give examples
% Cite and survey

% Define Co-Simulation and give examples
\cite{gomesCoSimulationSurvey2019}

% Define Coordination-framework and give examples
% Cite and survey?

% Maybe add the overview from Julien and me here to help with the definition

\section{Feature model} \label{sec:features}
% Cut up the feature model and describe it in dedicated subsections (maybe the first layer of nodes after the root)
% Maybe add a full feature model as an artifact (use GitHub repo with Zenodo)

\section{Classification examples} \label{sec:classifications}

% Classify some representatives, sticking to the same order as in the preliminaries.
\subsection{ADL example} % ADL representative
\subsection{Lingua Franca} % Coordination language representative
\subsection{Co-Simulation example} % Co-Simulation representative
\subsection{BCOOL} % Coordination framework representative
\subsection{BCorrLang} % Coordination framework representative --> Need to classify my approach for the thesis 

\section{Classification results}
% Clusters of features for ADL, Coordination language, Co-Simulation, and coordination frameworks?
% Maybe they can somehow be visualized nicely using some clusters with overlaps (vann diagram or something)

% Commonalities between clusters

% Differences between clusters

% What are the unique features of each approach? Does this match their intended usage scenarios?

\section{Running example}
% I'm not sure about this section yet, but a running example would be nice to show in different approaches.

\lstinputlisting[
label=lst:interactions,
language=Interaction,
caption=Interactions for the running example]{listings/interactions.txt}

\section{Conclusion} \label{sec:conclusion}

\bibliographystyle{splncs04} 
\bibliography{bib}

\end{document}
